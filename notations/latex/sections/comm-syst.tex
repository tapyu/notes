\section{Communication systems}
\subsection{Common symbols}
\begin{xltabular}{\textwidth}{XX}
	\(B\)                 & One-sided bandwidth of the baseband signal, in Hz                                 \\ \hline
	\(W\)                 & One-sided bandwidth of the baseband signal, in rad/s                              \\ \hline
	\(N_0\)               & Noise density, in ???                                                             \\ \hline
	\(x_i\)               & Real or in-phase part of \(x\)                                                    \\ \hline
	\(x_q\)               & Imaginary or quadrature part of \(x\)                                             \\ \hline
	\(f_c, f_{RF}\)       & Carrier frequency (in Hertz)                                                      \\ \hline
	\(f_L\)               & Carrier frequency in L-band (in Hertz)                                            \\ \hline
	\(f_{IF}\)            & Intermediate frequency (in Hertz)                                                 \\ \hline
	\(f_{s}\)             & Sampling frequency or sampling rate (in Hertz)                                    \\ \hline
	\(T_{s}\)             & Sampling time interval/duration/period                                            \\ \hline
	\(R\)                 & Bit rate                                                                          \\ \hline
	\(T\)                 & Bit interval/duration/period                                                      \\ \hline
	\(T_c\)               & Chip interval/duration/period                                                     \\ \hline
	\(T_{sy}, T_{sym}\)   & Symbol/signaling\cite{proakisDigitalCommunications2007} interval/duration/period  \\ \hline
	\(s_{RF}\)            & Transmitted signal in RF                                                          \\ \hline
	\(s_{FI}\)            & Transmitted signal in FI                                                          \\ \hline
	\(s, s_l\)            & Lowpass (or baseband) equivalent signal or envelope complex of transmitted signal \\ \hline
	\(r_{RF}\)            & Received signal in RF                                                             \\ \hline
	\(r_{FI}\)            & Received signal in FI                                                             \\ \hline
	\(r, r_l\)            & Lowpass (or baseband) equivalent signal or envelope complex of received signal    \\ \hline
	\(\phi\)              & Signal phase                                                                      \\ \hline
	\(\phi_0\)            & Initial phase                                                                     \\ \hline
	\(\eta_{RF}, w_{RF}\) & Noise in RF                                                                       \\ \hline
	\(\eta_{FI}, w_{FI}\) & Noise in FI                                                                       \\ \hline
	\(\eta, w\)           & Noise in baseband                                                                 \\ \hline
	\(\tau\)              & Timing delay                                                                      \\ \hline
	\(\Delta\tau\)        & Timing error (delay - estimated)                                                  \\ \hline
	\(\varphi\)           & Phase offset                                                                      \\ \hline
	\(\Delta\varphi\)     & Phase error (offset - estimated)                                                  \\ \hline
	\(f_d\)               & Linear Doppler frequency                                                          \\ \hline
	\(\Delta f_d\)        & Frequency error (Doppler frequency - estimated)                                   \\ \hline
	\(\nu\)               & Angular Doppler frequency                                                         \\ \hline
	\(\Delta \nu\)        & Frequency error (Doppler frequency - estimated)                                   \\ \hline
	\(\gamma, A\)         & Transmitted signal amplitude                                                      \\ \hline
	\(\gamma_0, A_0\)     & Combined effect of the path loss and antenna gain
\end{xltabular}
\subsection{Fading multipath channels}
\begin{xltabular}{\textwidth}{XX}
	\(t \overset{\mathcal{F}}{\leftrightarrow} \lambda\)  \cite{proakisDigitalCommunications2007}                                                                                                      & Support temporal of the signal. \(\lambda\) is obtained after taking the Fourier transform on \(t\).                                                                            \\ \hline
	\(\tau \overset{\mathcal{F}}{\leftrightarrow} f\)  \cite{proakisDigitalCommunications2007}                                                                                                         & Second support temporal of the signal (\(c(t)\) varies with with the input at the time \(\tau\)). \(f\) is obtained after taking the Fourier transform on \(\tau\).             \\ \hline
	\(c(t, \tau)\) \cite{proakisDigitalCommunications2007}                                                                                                                                             & Complex envelope of the channel response at the time \(t\) due to an impulse applied at the \(t - \tau\)                                                                        \\ \hline
	\(C(f,t)\) \cite{proakisDigitalCommunications2007}                                                                                                                                                 & Transfer function of \(c(t, \tau)\) in \(\tau\)                                                                                                                                 \\ \hline
	\(\alpha(t, \tau)\) \cite{proakisDigitalCommunications2007}                                                                                                                                        & Attenuation of \(c(t, \tau)\), i.e., \(c(t, \tau) = \alpha(t, \tau) e^{e\pi f_c \tau}\)                                                                                         \\ \hline
	\(R_c(\tau_1, \tau_2, \Delta t)\) \cite{proakisDigitalCommunications2007}                                                                                                                          & Autocorrelation function of \(c(t, \tau)\), i.e., \(R_c(\tau_1, \tau_2, \Delta t) = \E{c^*(t, \tau_1), c^*(t + \Delta t, \tau_2)}\)                                             \\ \hline
	\(R_c(\tau, \Delta t)\) \cite{proakisDigitalCommunications2007}                                                                                                                                    & Autocorrelation function of \(c(t, \tau)\) assuming uncorrelated scattering                                                                                                     \\ \hline
	\(R_c(\tau), \eval{R_c(\tau, \Delta t)}_{\Delta t = 0}\) \cite{proakisDigitalCommunications2007}                                                                                                   & Multipath intensity profile or delay power spectrum                                                                                                                             \\ \hline
	\(R_C(\Delta f, \Delta t), R_C(f_1, f_2; \Delta t)\), \(\E{C(f_1,t), C(f_2, t + \Delta t)}\), \(\mathcal{F}_\tau \left\{ R_c(\tau, \Delta t) \right\}\) \cite{goldsmithWirelessCommunications2005} & Spaced-frequency, spaced-time correlation function (\(\Delta f = f_2 - f_1\))                                                                                                   \\ \hline
	\(R_C(\Delta f)\), \(\eval{R_C(\Delta f, \Delta t)}_{\Delta t = 0}\) \cite{proakisDigitalCommunications2007}, \(\mathcal{F}\left\{ R_c(\tau) \right\}\) \cite{goldsmithWirelessCommunications2005} & Spaced-frequency correlation function                                                                                                                                           \\ \hline
	\((\Delta f)_c\)                                                                                                                                                                                   & Coherence bandwidth of \(c(t)\), that is, the frequency interval in which \(R_C(\Delta f)\) is nonzero \cite{proakisDigitalCommunications2007}                                  \\ \hline
	\(T_m\)                                                                                                                                                                                            & Multipath spread of the channel, that is, the time interval in which \(R_c(\tau)\) is nonzero (\(T_m \approx 1/(\Delta f)_c \)) \cite{proakisDigitalCommunications2007}         \\ \hline
	\(R_C(\Delta t), \eval{R_C(\Delta f, \Delta t)}_{\Delta f = 0}\)                                                                                                                                   & Spaced-time correlation function \cite{proakisDigitalCommunications2007}                                                                                                        \\ \hline
	\(S_C(\lambda)\) \cite{proakisDigitalCommunications2007}, \(\mathcal{F}\left\{ R_C (\Delta t) \right\}\) \cite{goldsmithWirelessCommunications2005}                                                & Doppler power spectrum                                                                                                                                                          \\ \hline
	\((\Delta t)_c\)                                                                                                                                                                                   & Coherence time of \(c(t)\), that is, the time interval in which \(R_C(\Delta t)\) is nonzero \cite{proakisDigitalCommunications2007}                                            \\ \hline
	\(B_m\)                                                                                                                                                                                            & Multipath spread of the channel, that is, the frequency interval in which \(S_c(\lambda)\) is nonzero (\(B_d \approx 1/(\Delta t)_c \)) \cite{proakisDigitalCommunications2007} \\ \hline
	\(S_C(\tau, \lambda)\) \cite{proakisDigitalCommunications2007}, \(\mathcal{F}_{\Delta f, \Delta t}\left\{ R_C (\Delta f, \Delta t) \right\}\) \cite{goldsmithWirelessCommunications2005}           & Scattering function                                                                                                                                                             \\ \hline
\end{xltabular}
